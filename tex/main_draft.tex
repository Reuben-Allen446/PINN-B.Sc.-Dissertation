
\documentclass[12pt,a4paper]{article}
\usepackage[utf8]{inputenc}
\usepackage[british]{babel}
\usepackage{amsmath,amssymb}
\usepackage{graphicx}
\usepackage[numbers]{natbib}
\usepackage{setspace}
\usepackage{hyperref}
\usepackage{geometry}
\geometry{margin=1in}
\onehalfspacing

\title{Are Physics-Informed Neural Networks the Future of Financial Modelling? (Draft)}
\author{Reuben Allen \\ 10824401 \\ The University of Manchester}
\date{\today}

\begin{document}

\maketitle

\begin{abstract}
Physics-Informed Neural Networks (PINNs) incorporate physical laws into neural network training, providing a flexible framework for solving PDEs. This dissertation investigates their performance on classical wave equations and financial PDEs like the Black–Scholes model. It examines their robustness under nonlinear conditions, periodic boundaries, and stochastic elements. Preliminary results show PINNs excel in adaptability and generalisation but highlight challenges in training stability and computational efficiency.
\end{abstract}

\newpage
\tableofcontents
\newpage

\section{Introduction}

Partial differential equations (PDEs) are essential tools in physics and finance, underpinning models from fluid dynamics to option pricing~\cite{black1973pricing}. Traditional numerical solutions like finite difference methods (FDM) are robust but limited in adaptability and computational cost~\cite{leveque2007finite,morton2005numerical}. Physics-Informed Neural Networks (PINNs) are an emerging mesh-free approach that integrates PDE constraints directly into neural networks~\cite{raissi2019physics,karniadakis2021physics}, showing promise in various disciplines, including finance~\cite{sirignano2018deep,cuomo2022scientific}. This dissertation evaluates PINNs' ability to generalise, handle complexity, and outperform traditional methods under dynamic conditions.

\section{Mathematical Background}

\subsection{The Classical Wave Equation}

The wave equation describes various physical phenomena:
\[
\frac{\partial^2 u}{\partial t^2}=c^2\nabla^2 u.
\]
Its higher-dimensional form tests PINNs' scalability~\cite{cuomo2022scientific,raissi2019physics}.

\subsection{Boundary Conditions}

Boundary conditions define PDE solutions, with periodic conditions particularly relevant to financial cycles~\cite{zhu2022periodic}.

\subsection{Nonlinear Wave Equations}

Adding a nonlinear term tests PINNs' adaptability:
\[
\frac{\partial^2 u}{\partial t^2}=c^2\nabla^2 u-\beta u^3.
\]

\section{Classical PINNs (Notebook A)}

PINNs minimise a composite loss function incorporating PDE residuals, initial conditions, and boundary conditions~\cite{raissi2019physics,baydin2018automatic}. Preliminary results indicate excellent agreement with analytical solutions in 1D and 3D cases.

\textit{(Insert here: Loss convergence plots, 3D wave surface visualisations.)}

\section{Nonlinear PINNs and Generalisation (Notebook A)}

Nonlinear PDEs introduce optimisation complexity~\cite{wang2022understanding}. Periodic boundary conditions significantly improve extrapolation and stability~\cite{karniadakis2021physics}.

\textit{(Insert here: Nonlinear loss curves, periodic vs classical error plots.)}

\section{PINNs in Financial Modelling (Notebook B)}

\subsection{Black–Scholes Equation}

PINNs efficiently solve the classical Black–Scholes PDE, demonstrating mesh-free flexibility:
\[
\frac{\partial V}{\partial t}+\frac{1}{2}\sigma^2S^2\frac{\partial^2 V}{\partial S^2}+rS\frac{\partial V}{\partial S}-rV=0.
\]

\textit{(Insert here: Predicted pricing surfaces.)}

\subsection{Stochastic Volatility}

Introducing stochastic volatility via PINNs captures market realism and uncertainty~\cite{zang2020adaptive,yang2021bpn}.

\textit{(Insert here: Stochastic volatility surfaces, uncertainty quantification results.)}

\subsection{Real-World Financial Data}

Hybrid approaches combining empirical data with PINNs offer improved accuracy in practical scenarios, despite challenges balancing theoretical and empirical constraints~\cite{finlay2022train}.

\textit{(Insert here: Error maps near discontinuities.)}

\section{Comparison with Traditional Methods}

Table~\ref{tab:comparison} compares traditional solvers, classical PINNs, and nonlinear/periodic PINNs in terms of flexibility, generalisation, and computational cost.

\textit{(Insert here: Performance comparison table.)}

\section{Discussion}

PINNs' PDE-driven regularisation yields strong generalisation within training domains~\cite{cuomo2022scientific}. However, nonlinearity poses challenges, mitigated by periodic constraints~\cite{zhu2022periodic}. Computational demands and training stability are ongoing concerns, balanced by PINNs' flexibility and adaptability to stochastic processes~\cite{zang2020adaptive,mishra2022estimates}.

\section{Conclusion}

PINNs demonstrate significant potential in financial modelling, particularly for adaptable, complex PDEs. Future research directions, including quantum-enhanced PINNs and hybrid data-driven models, promise further advances~\cite{zhang2022quantum,mcardle2020quantum}.

\newpage
\section*{Appendices}

\subsection*{Appendix A: Notebook A Visualisations}

- Classical and nonlinear wave equation results.
- Periodic boundary condition improvement.

\subsection*{Appendix B: Notebook B Visualisations}

- Black–Scholes pricing surfaces.
- Stochastic volatility predictions.

\subsection*{Appendix C: Performance Comparison Table}

- Comparative analysis of traditional methods and PINNs.

\newpage
\bibliographystyle{plainnat}
\bibliography{pinn_references_draft}

\end{document}
